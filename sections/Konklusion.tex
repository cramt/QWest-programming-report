% Conclusion
% What did we learn?
% What was a success? To what degree?
% What went not as well?
\chapter{Konklusion}\label{ch:conclusion}
Målet med projektet var at udvikle et distribueret system som kunne bruges til at dokumentere rejseoplevelser, og skulle håndtere samtidighed mellem brugere.
Det lykkedes i dette projekt at designe det distribuerede system med fokus på services, som skulle kommunikere med hinanden gennem HTTP eller SQLConnections. Fordelene ved at uddeligere forskellige opgaver og dele af systemet til forskellige services, gjorde at koblingen blev lav, hvilket er fordelagtigt til videreudvikling og vedligehold af systemet. For at tilgå systemet kunne det lade sig gøre enten som bruger eller administrator, hvor der var forskellige muligheder for hvad man havde tilladelse til at gøre. 
Arkitekturen skulle så implementeres, ved brug af C\# kode til backend, Model klasser og Database adgang, og HTML, CSS og JavaScript til front-end og web. Selve udviklingen af systemet blev gennemført med TDD, som sikrede at alle metoder, der skulle bruges til at hente og lagre data i databasen, samt vores API virkede som de skulle før de blev implementeret. Samtlige tests blev gennemført korrekt og bestod. 
Samtidighedsproblemet i dette projekt opstod, når flere brugere fra samme gruppe ville redigere et gruppe-opslag samtidigt. Løsningen til dette problem var en optimistisk samtidighedsløsning, hvor alle gruppemedlemmerne kan tilgå opslaget samtidig, men hvor den seneste ændring der gemmes er den som bliver lagret i databasen når alle gruppemedlemmerne har foretaget deres ændringer. Dette blev valgt fordi den optimistiske tilgang passede bedst til systemet der er bygget på REST principperne. Skulle der være implementeret en pessimistisk løsning, skulle der bruges konstante forbindelser mellem database og system, hvilket går imod REST principperne.
Alle de centrale features der skulle implementeres blev implementeret sådan at de også virker korrekt, og dette projekt er derfor en god løsning til projektets problemformulering. 