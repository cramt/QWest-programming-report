\ucntitlepage{%
\parbox[t]{\titlepageleftcolumnwidth}{
  \danishprojectinfo{
    QWest %title - update this
  }{%
    5/10/20 - 21/12/20 %project period
  }{%
    Gruppe 5 % project group
  }{%
    %list of group members
    Benyad Jomhur\\ 
    Lucca Christiansen\\
    Alexandra Østermark\\
    Søren Ravn\\
  }{%
    %list of supervisors
    Lars Nysom\\
    Michael Holm Andersen\\
    Henrik Kristian Ulrik Øllgaard\\
  }{%
    \charactercount
  }{%
    \pagecount
  }{%
    \url{https://github.com/cramt/QWest}
  }{
    \url{URL_TO_COMMIT_ON_REPO}  %TODO: Update this on the 20th or 21st
  }
  \danishprojectinfocont{%
    \today % date of completion
  }%
  }
}{%department and address
  \textbf{IT-uddannelserne}\\
  Professionshøjskolen UCN\\
  \href{http://www.ucn.dk}{www.ucn.dk}
}{% the abstract
The following report provides a detailed explanation of the Qwest system application. We'll see throughout the report how we've solved issues, such as choosing the correct design, implementing, and how the concurrency problem was managed. The report digs down to the core of the architecture, showing and explaining the system architecture in a simple and understandable way. The architecture section of the report has been an large focus since it includes how we've managed to create a distributed system, based on the requirements. It includes services such as RESTful Web API, and other important blueprint requirements, like Database design and Web design to get the program running as smoothly as possible. As the program displays a map of the world too which will be discussed exactly how, that it is we've managed to encrypt locations, in the implementation section. Not only will location encryptions be discussed here, but the implementation of the Database, DAO, Services, Web, and at last how testing was carried out.
