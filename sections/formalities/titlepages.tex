\ucntitlepage{%
\parbox[t]{\titlepageleftcolumnwidth}{
  \danishprojectinfo{
    QWest %title - update this
  }{%
    5/10/20 - 21/12/20 %project period
  }{%
    Gruppe 5 % project group
  }{%
    %list of group members
    Benyad Jomhur\\ 
    Lucca Christiansen\\
    Alexandra Østermark\\
    Søren Ravn\\
  }{%
    %list of supervisors
    Lars Nysom\\
    Michael Holm Andersen\\
    Henrik Kristian Ulrik Øllgaard\\
  }{%
    \charactercount
  }{%
    \pagecount
  }{%
    \url{https://github.com/cramt/QWest}
  }{
    \url{URL_TO_COMMIT_ON_REPO}  %TODO: Update this on the 20th or 21st
  }
  \danishprojectinfocont{%
    \today % date of completion
  }%
  }
}{%department and address
  \textbf{IT-uddannelserne}\\
  Professionshøjskolen UCN\\
  \href{http://www.ucn.dk}{www.ucn.dk}
}{% the abstract
The following report provides a detailed explanation of the implementation and execution of the Qwest application. The purpose of QWest is being a platform where one can check off countries on a progress map, and share experiences with friends and groups through pictures, descriptions and locations. Throughout the report it is discussed how different issues such as choosing the correct design and implementing the solutions were handled, as well as how the concurrency problem was managed. The report shows how the system was design by showing and explaining the system architecture with diagrams, and later implementation details. In the the architecture section it is explained how we've managed to design a distributed system, based on given requirements. It includes services such as RESTful Web API, and other important blueprint requirements, like Database design and Web design with a focus on decoupling, security and communication. The implementation section focuses on code examples showing implementation details, which are further explained with regards to communication, middleware and concurrency issues. There will be plenty of buzzwords such as Database, DAO, Services, Web, which will be explained and explored. Finally it was concluded that the program in it's delivered state was a success as all central features, including the concurrency problem, were implemented and working.
}